\documentclass{article}
\usepackage{cv}

\title{Curriculum Vitae}
\author{Mauricio Menezes}

\begin{document}

% CV Header section
{\huge{\color{slateblue}\textbf{Mauricio Souza Menezes}}} \\
\rule{\textwidth}{0.5mm} \\

% Personal Details section
\parbox{0.5\textwidth}{
	\begin{tabbing}
		\hspace{3cm} \= \hspace{4cm} \= \kill
		{\bf Endereço} \> Rua João Gonçalves Tourinho, \\
		\> Rio Vermelho, 39, Salvador/Bahia \\
		{\bf Data de } \\
		{\bf Nascimento} \> 30 de Julho de 1995 \\
		{\bf Nacionalidade} \> Brasileiro
	\end{tabbing}
}\hfil\parbox{0.5\textwidth}{
	\begin{tabbing}
		\hspace{3cm} \= \hspace{4cm} \= \kill

		{\bf Telefone}
		\> +55 (71) 99241 4527 \\

		{\bf Email} \>
		\href{mailto:mauriciosm95@gmail.com}{mauriciosm95@gmail.com} \\

		{\bf LinkedIn} \>
		\href{https://www.linkedin.com/in/mau-me/}{linkedin.com/in/mau-me} \\

		{\bf Github} \>
		\href{https://github.com/mau-me}{github.com/mau-me} \\

		{\bf Portfolio} \>
		\href{https://mau-me.github.io/maume_page/}{mau-me.github.io} \\
	\end{tabbing}
}

% % Personal profile section
% \section*{Personal Profile}

% Lorem ipsum dolor sit amet, consectetur adipiscing elit. Duis elementum nec dolor sed sagittis. Cras justo lorem, volutpat mattis lacus vel, consequat aliquam quam. Interdum et malesuada fames ac ante ipsum primis in faucibus. Integer blandit, massa at tincidunt ornare, dolor magna interdum felis, ac blandit urna neque in turpis.

% Education section
\section*{Formação Acadêmica}

\begin{tabbing}
	\hspace{2cm} \= \hspace{4cm} \= \kill
	\bf{2025--2027} \> Mestrando em Ciência da Computação -\ Inteligência Computacional e Otimização \\
	\href{https://www.ufba.br/}{Universidade Federal da Bahia} \\
	\href{https://pgcomp.ufba.br/}{Programa de Pós-Graduação em Ciência da Computação} \\
\end{tabbing}

\begin{tabbing}
	\hspace{2cm} \= \hspace{4cm} \= \kill
	\bf{2013--2023} \> Bacharel em Sistemas de Informação \\
	\href{https://portal.uneb.br/}{Universidade do Estado da Bahia} \\
\end{tabbing}

% Employment History section
\section*{Experiência}

\begin{job}
	{07/2024 -}{Atual}
	{Renova Soluções em Tecnologia}
	{https://renova.net.br/site/}
	{Desenvolvedor}%
	{Desenvolvimento e suporte de sistemas, além de realizar a definição da arquitetura de sistemas, bem como ser responsável por desenvolver e implementar padrões de projetos, aplicados em toda a empresa, além de instruir e orientar no setor de desenvolvimento e DevOps.
		\rule{0mm}{5mm}\textbf{Tecnologias:} C\#, {.Net}
		Core/Framework, Asp.Net, Nhibernate, Entity Framework, Oracle
		DB, SQL Server.}
\end{job}

\begin{job}
	{02/2019 -}{01/2020}
	{Sinqia}
	{https://sinqia.com.br/}
	{Desenvolvedor}%
	{Desenvolvimento e suporte de sistemas, além de criar e analisar consultas, stored procedures, functions e views em bases de dados.\\
		\rule{0mm}{5mm}\textbf{Tecnologias:} C\#, {.Net}
		Core/Framework, Asp.Net, Nhibernate, Entity Framework, Oracle
		DB, SQL Server.}
\end{job}

\begin{job}
	{01/2017 -}{01/2018}
	{Universidade do Estado da Bahia}
	{https://portal.uneb.br/}
	{Desenvolvedor / Bolsista de Monitoria de Extensão}%
	{Criação e manutenção do portal de apoio a publicações científicas do departamento de ciências exatas e da terra.\\
		\rule{0mm}{5mm}\textbf{Tecnologias:} HTML, CSS, Javascript e
		MySQL}
\end{job}

\begin{job}
	{01/2017 -}{01/2018}
	{Universidade do Estado da Bahia}
	{https://portal.uneb.br/}
	{DBA}%
	{Criação, instalação, monitoramento, reparo, análise e suporte no uso de ferramentas de banco de dados.\\
		\rule{0mm}{5mm}\textbf{Tecnologias:} MySQL}
\end{job}

% Education section
\section*{Certificações}

\begin{tabbing}
	\hspace{2cm} \= \hspace{4cm} \= \kill
	\bf{2023} \> Engenharia de Dados AWS
	-~\href{https://howedu.com.br/}{How Education} \\
	Amazon Web Service (AWS), Airflow, Cloudformation, Data Lakes, Docker, Python, PyTest, Terraform, Integração Contínua (GitHub\\ Actions), SQL, API Rest, Selenium, Jenkins. \\
\end{tabbing}

\begin{tabbing}
	\hspace{2cm} \= \hspace{4cm} \= \kill
	\bf{2023} \> Google Cloud Computing Foundations
	-~\href{https://www.cloudskillsboost.google/public_profiles/3a03b48f-00f0-42a3-9800-4c045d78b21e/badges/3540421}{Google
		Cloud} \\
	Cloud Computing, Infraestrutura, Redes e Segurança, Dados, Machine Learning e Inteligencia Artificial\\ em Google Cloud. \\
\end{tabbing}

\begin{tabbing}
	\hspace{2cm} \= \hspace{4cm} \= \kill
	\bf{2023} \> Desenvolvimento de Software (Foco em Backend)
	-~\href{https://cubos.academy/}{Cubos Academy} \\
	Git e GitHub, Javascript, Node.JS, SQL, HTTP, API Rest, Streaming e Teste de Software. \\
\end{tabbing}

\begin{tabbing}
	\hspace{2cm} \= \hspace{4cm} \= \kill
	\bf{2021--2022} \> Tecnologia da Informação (NuLab)

	-~\href{https://www.alura.com.br/}{Alura} e
	\href{https://nubank.com.br/}{Nubank} \\

	Desenvolvimento Back-end, Node.JS, API's Rest, PostgreSQL, Git e GitHub \\
	Metodologias Ágeis e SoftSkills. \\
\end{tabbing}

% IT/Computing Skills section
\section*{Software Skills}

\begin{skillgroup}{Linguagens de Programação}%
	% \skill{C\#/{.}Net Core/Framework} \\
	\skill{.Net Core/Framework (C\#)} \\
	\skill{Java} \\
	\skill{Javascript/Node.JS} \\
	\skill{Python}
\end{skillgroup}

\begin{skillgroup}{Desenvolvimento Web}%
	\skill{ASP.NET} \\
	\skill{HTML5, CSS3, JavaScript} \\
	\skill{React} \\
	\skill{Express.JS}
\end{skillgroup}

\begin{skillgroup}{Cloud}%
	\skill{Google Cloud} -\ google cloud skills boost\\
	\skill{AWS} \\
	\skill{Azure}
\end{skillgroup}

\begin{skillgroup}{Banco de Dados}%
	\skill{MySQL} \\
	\skill{PostgreSQL} \\
	\skill{SQL Server} \\
	\skill{MongoDB} \\
	\skill{Oracle Database} \\
	\skill{Redis}
\end{skillgroup}

\begin{skillgroup}{Outros}%
	\skill{Apache/Nginx} \\
	\skill{Docker} \\
	\skill{Git} \\
	\skill{GitHub} \\
	\skill{CI/CD -\ Azure DevOps, GitHub Actions} \\
	\skill{Linux} \\
	\skill{Pytest} \\
	\skill{Selenium} \\
	\skill{API Rest} \\
	\skill{Metodologias Ágeis -\ Scrum Kanban} \\
	\skill{Inteligencia Artificial -\ Machine Learning}
\end{skillgroup}

% Interests section
% \section*{Interesses}

% \begin{tabbing}
% 	\hspace{5mm} \= \kill
% 	\rule{0mm}{4mm}\sqbullet\> \textbf{Corrida, Futebol, Natação} \\
% 	\rule{0mm}{4mm}\sqbullet\> \textbf{Travelling} \\
% 	\rule{0mm}{4mm}\sqbullet\> \textbf{Livros de TI e Gerais} \\
% 	% \rule{0mm}{4mm}\sqbullet\> \textbf{Car Mechanics} \\

% \end{tabbing}

% % References section
% \section*{Referees}

% Available on request.

\end{document}
