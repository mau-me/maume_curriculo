\documentclass{article}
\usepackage{cv}

\title{Curriculum Vitae}
\author{Mauricio Menezes}

\begin{document}

% CV Header section
{\huge{\color{slateblue}\textbf{Mauricio Souza Menezes}}} \\
\rule{\textwidth}{0.5mm} \\

% Personal Details section
\parbox{0.5\textwidth}{
	\begin{tabbing}
		\hspace{3cm} \= \hspace{4cm} \= \kill
		{\bf Endereço} \> Rua João Gonçalves Tourinho, \\
		\> Rio Vermelho, 39, Salvador/Bahia \\
		{\bf Data de } \\
		{\bf Nascimento} \> 30 de Julho de 1995 \\
		{\bf Nacionalidade} \> Brasileiro
	\end{tabbing}
}\hfil\parbox{0.5\textwidth}{
	\begin{tabbing}
		\hspace{3cm} \= \hspace{4cm} \= \kill

		{\bf Telefone}
		\> +55 (71) 99241 4527 \\

		{\bf Email} \>
		\href{mailto:mauriciosm95@gmail.com}{mauriciosm95@gmail.com} \\

		{\bf LinkedIn} \>
		\href{https://www.linkedin.com/in/mau-me/}{linkedin.com/in/mau-me} \\

		{\bf Github} \>
		\href{https://github.com/mau-me}{github.com/mau-me} \\

		{\bf Portfolio} \>
		\href{https://mau-me.github.io/}{mau-me.github.io} \\
	\end{tabbing}
}

% Personal profile section
\section*{Perfil Profissional}

Desenvolvedor Backend com mais de 5 anos de experiência em desenvolvimento de software, atuando principalmente com a construção de sistemas escaláveis e de alta performance. Especializado em linguagens como C\#, Python e Javascript, com forte conhecimento em frameworks como {.NET} Core e Node.js. Experiência significativa em desenvolvimento de APIs RESTful, implementação de microsserviços e design patterns. Habilidade em trabalhar com bancos de dados relacionais como PostgreSQL, SQL Server, MySQL e Oracle, além de bancos NoSQL como {MongoDB} e Redis. Experiência em ambientes de nuvem como AWS e Google Cloud, com foco em práticas de DevOps e integração contínua. Capacidade comprovada de liderar equipes técnicas, mentorando desenvolvedores e promovendo boas práticas de desenvolvimento. Conhecimento em machine learning, llm`s e otimização de algoritmos, com experiência acadêmica na área. Proficiente em metodologias ágeis como Scrum e Kanban, com forte ênfase em testes automatizados e integração contínua (CI/CD).

% Education section
\section*{Formação Acadêmica}

\begin{tabbing}
	\hspace{2cm} \= \hspace{4cm} \= \kill
	\bf{2025--2027} \> Mestrando em Ciência da Computação -\ Inteligência Computacional e Otimização \\
	\href{https://www.ufba.br/}{Universidade Federal da Bahia} \\
	\href{https://pgcomp.ufba.br/}{Programa de Pós-Graduação em Ciência da Computação} \\
	\> \textit{Área de pesquisa: Otimização de algoritmos aplicados a instalação de usinas eólicas} \\
\end{tabbing}

\begin{tabbing}
	\hspace{2cm} \= \hspace{4cm} \= \kill
	\bf{2013--2023} \> Bacharel em Sistemas de Informação \\
	\href{https://portal.uneb.br/}{Universidade do Estado da Bahia} \\
	\> \textit{{TCC}: Implementação de um modelo de machine learning para análise} \\
	\> \textit{de sequências virais do SARS-COV-2} \\
\end{tabbing}

% Employment History section
\section*{Experiência Profissional}

\begin{job}
{07/2024 -}{Atual}
{Renova Soluções em Tecnologia}
{https://renova.net.br/site/}
{Desenvolvedor Backend}%
{Desenvolvimento e manutenção de aplicações empresariais de alta escalabilidade, com as seguintes responsabilidades:
	\begin{itemize-noindent}
	\item Definição e implementação de arquitetura de sistemas baseada em microsserviços
	\item Desenvolvimento e implementação de padrões de projetos (Design Patterns) aplicados em toda a empresa
	\item Mentoria técnica e orientação para equipes de desenvolvimento e DevOps
	\item Otimização de consultas e procedimentos em bancos de dados relacionais
	\item Implementação de práticas de CI/CD e testes automatizados
	\end{itemize-noindent}
	\rule{0mm}{5mm}\textbf{Tecnologias principais:} C\#, {.Net Core}/Framework, ASP.NET Web API, Nhibernate, Entity Framework, Oracle DB, SQL Server, Azure DevOps.}
\end{job}

\begin{job}
{02/2019 -}{01/2020}
{Sinqia}
{https://sinqia.com.br/}
{Desenvolvedor Backend}%
{Atuação em projetos para o setor financeiro, com foco em:
	\begin{itemize-noindent}
	\item Desenvolvimento de APIs RESTful seguindo padrões de mercado
	\item Criação e otimização de consultas SQL complexas, stored procedures e functions
	\item Implementação de regras de negócio para sistemas de processamento bancário
	\item Participação em revisões de código e melhorias de arquitetura
	\end{itemize-noindent}
	\rule{0mm}{5mm}\textbf{Tecnologias principais:} C\#, {.Net Core}/Framework, ASP.NET MVC, Nhibernate, Entity Framework, Oracle DB, SQL Server.}
\end{job}

\begin{job}
{01/2017 -}{01/2018}
{Universidade do Estado da Bahia}
{https://portal.uneb.br/}
{Desenvolvedor Web / Bolsista de Monitoria de Extensão}%
{Desenvolvimento do portal de apoio a publicações científicas do departamento:
	\begin{itemize-noindent}
	\item Implementação de sistema completo de cadastro e gestão de publicações acadêmicas
	\item Criação de interface responsiva para acesso em dispositivos móveis
	\item Integração com sistemas internos da instituição
	\end{itemize-noindent}
	\rule{0mm}{5mm}\textbf{Tecnologias:} HTML5, CSS3, JavaScript, PHP, MySQL}
\end{job}

\begin{job}
{01/2017 -}{01/2018}
{Universidade do Estado da Bahia}
{https://portal.uneb.br/}
{Administrador de Banco de Dados (DBA)}%
{Responsável pela infraestrutura de banco de dados do departamento:
	\begin{itemize-noindent}
	\item Instalação, configuração e otimização de servidores MySQL
	\item Implementação de rotinas de backup e recuperação de dados
	\item Monitoramento de performance e resolução de problemas
	\item Criação de relatórios analíticos para tomada de decisão
	\end{itemize-noindent}
	\rule{0mm}{5mm}\textbf{Tecnologias:} MySQL, Shell Script, Ferramentas de monitoramento}
\end{job}

% Projetos section
\section*{Projetos Relevantes}

\begin{tabbing}
	\hspace{2cm} \= \hspace{4cm} \= \kill
	\bf{2023 -\ 2024} \> \textbf{API de Processamento de Dados Financeiros} \\
	\> Desenvolvimento de uma API RESTful para processamento de transações financeiras \\
	\> com alta performance e baixa latência. Implementação de padrões como Repository, \\
	\> Factory e Strategy. \textbf{Tecnologias:} {.NET Core}, SQL Server, Redis, Docker. \\
\end{tabbing}

\begin{tabbing}
	\hspace{2cm} \= \hspace{4cm} \= \kill
	\bf{2022 -\ 2023} \> \textbf{Sistema de Análise Preditiva} \\
	\> Implementação de algoritmos de machine learning para análise preditiva de dados. \\
	\> \textbf{Tecnologias:} Python, Pandas, Scikit-learn, Docker, {MongoDB}. \\
\end{tabbing}

% Certificações section
\section*{Certificações}

\begin{tabbing}
	\hspace{2cm} \= \hspace{4cm} \= \kill
	\bf{2023} \> \textbf{Engenharia de Dados AWS} -~\href{https://howedu.com.br/}{How Education} \\
	\> Amazon Web Service (AWS), Airflow, CloudFormation, Data Lakes, Docker, Python, PyTest, \\
	\> Terraform, CI/CD (GitHub Actions), SQL, API Rest, Selenium, Jenkins. \\
\end{tabbing}

\begin{tabbing}
	\hspace{2cm} \= \hspace{4cm} \= \kill
	\bf{2023} \> \textbf{Google Cloud Computing Foundations} -~\href{https://www.cloudskillsboost.google/public_profiles/3a03b48f-00f0-42a3-9800-4c045d78b21e/badges/3540421}{Google Cloud} \\
	\> Cloud Computing, Infraestrutura, Redes e Segurança, Dados, Machine Learning e Inteligência \\
	\> Artificial em Google Cloud. \\
\end{tabbing}

\begin{tabbing}
	\hspace{2cm} \= \hspace{4cm} \= \kill
	\bf{2023} \> \textbf{Desenvolvimento de Software (Backend)} -~\href{https://cubos.academy/}{Cubos Academy} \\
	\> Git e GitHub, JavaScript, Node.JS, SQL, HTTP, API Rest, Streaming e Teste de Software. \\
\end{tabbing}

\begin{tabbing}
	\hspace{2cm} \= \hspace{4cm} \= \kill
	\bf{2021--2022} \> \textbf{Tecnologia da Informação (NuLab)} -~\href{https://www.alura.com.br/}{Alura} e \href{https://nubank.com.br/}{Nubank} \\
	\> Desenvolvimento Backend, Node.JS, APIs RESTful, PostgreSQL, Git e GitHub, \\
	\> Metodologias Ágeis e Soft Skills. \\
\end{tabbing}

% IT/Computing Skills section
\section*{Competências Técnicas}

\begin{skillgroup}{Backend \& Arquitetura}%
	\skill{Python} -\ Avançado \\
	\skill{Django / Flask / FastAPI} -\ Avançado \\
	\skill{.NET Core/Framework (C\#)} -\ Avançado \\
	\skill{ASP.NET Web API / MVC} -\ Avançado \\
	\skill{Design Patterns} -\ Avançado \\
	\skill{Node.JS} -\ Avançado \\
	\skill{Java} -\ Intermediário \\
	\skill{Microsserviços} -\ Intermediário \\
\end{skillgroup}

\begin{skillgroup}{Bancos de Dados}%
	\skill{SQL Server} -\ Avançado \\
	\skill{Oracle Database} -\ Avançado \\
	\skill{MySQL} -\ Avançado \\
	\skill{PostgreSQL} -\ Avançado \\
	\skill{MongoDB} -\ Avançado \\
	\skill{Redis} -\ Intermediário
\end{skillgroup}

\begin{skillgroup}{DevOps \& Cloud}%
	\skill{Git / GitHub} -\ Avançado \\
	\skill{CI/CD -\ Azure DevOps, GitHub Actions} -\ Avançado \\
	\skill{Docker} -\ Avançado \\
	\skill{AWS} -\ Intermediário \\
	\skill{Google Cloud} -\ Intermediário \\
	\skill{Azure} -\ Intermediário \\
	\skill{Linux} -\ Avançado
\end{skillgroup}

\begin{skillgroup}{Desenvolvimento Web}%
	\skill{API RESTful} -\ Avançado \\
	\skill{HTML5, CSS3, JavaScript} -\ Avançado \\
	\skill{React} -\ Intermediário \\
	\skill{Express.JS} -\ Avançado
\end{skillgroup}

\begin{skillgroup}{Inteligência Artificial}%
	\skill{Modelos de Machine Learning} -\ Avançado \\
	\skill{Modelos de LLMs (Large Language Models)} -\ Avançado \\
	\skill{Visão Computacional} -\ Intermediário \\
	\skill{Otimização de Algoritmos} -\ Intermediário
	\skill{Processamento de Linguagem Natural (NLP)} -\ Intermediário \\
\end{skillgroup}

\begin{skillgroup}{Metodologias \& Práticas}%
	\skill{Scrum / Kanban} -\ Avançado \\
	\skill{Testes Automatizados (PyTest, Selenium)} -\ Avançado \\
	\skill{Integração Contínua} -\ Avançado \\
	\skill{Clean Code / SOLID} -\ Avançado \\
\end{skillgroup}

% Idiomas section
\section*{Idiomas}

\begin{tabbing}
	\hspace{2cm} \= \hspace{4cm} \= \kill
	\bf{Português} \> Nativo \\
	\bf{Inglês} \> Intermediário -\ Leitura técnica e escrita
\end{tabbing}

\end{document}